\subsection{The Poisson structure $\mathcal{F}_*(\pi_{\bg}^{\dagger})$}
Let $\bg:=(\Gamma_1,\Gamma_2,\gamma)$ be a BD triple of type $A_{n-1}$. Define a rational map $\mathcal{C}:\GL_n\dashrightarrow \GL_n$ via
\begin{equation}
    \mathcal{C}(U):=U\cdot \rho(U) = U \prod_{k=1}^{\rightarrow} \tilde{\gamma}^*(U_-), \ \ U \in \GL_n.
\end{equation}
The map $\mathcal{C}$ is in fact birational, with the inverse given by
\begin{equation}\label{eq:cinv}
    \mathcal{C}^{-1}(U) = U\cdot \tilde{\gamma}^*(U_-)^{-1}, \ \ U \in \GL_n.
\end{equation}
Set $\pi_{\mathcal{F}}:= \mathcal{F}_*(\pi_{\bg}^{\dagger})$. Since $\mathcal{F}^c(U) = \mathcal{F}(U) \tilde{\gamma}^*(\mathcal{F}(U)_-)^{-1}$, the following diagram is commutative:
\begin{equation}
\xymatrix{
    (\GL_n,\pi_{\bg}^{\dagger})\ar@{-->}[d]_{\mathcal{F}} \ar@{-->}[r]^{\mathcal{F}^c}  & (\GL_n,\pi_{\std}^{\dagger}) \ar@{-->}[ld]^{\mathcal{C}} \\
     (\GL_n,\pi_{\mathcal{F}})  &
    }
\end{equation}
Moreover, all the arrows are birational Poisson isomorphisms (provided the $r_0$-parts are the same for all Poisson bivectors). The Poisson bracket $\{\cdot , \cdot \}_{\mathcal{F}}$ that corresponds to $\pi_{\mathcal{F}}$ is given by
\begin{equation}\begin{split}
    \{f,g\}_{\mathcal{F}} = &\langle R_0\pi_0[U,\nabla_Uf],[U,\nabla_U g]\rangle + \langle \pi_0[U,\nabla_Uf],\nabla_U^L g\rangle + \\ + &\langle \pi_{>} \nabla^L_Uf,\nabla_U^L g\rangle - \langle \pi_{>}\nabla_U^Rf,\nabla_U^Rg\rangle+\\ + &\langle \frac{1}{1-\gamma}\pi_{>}\nabla_U^R f,\nabla_U^R g\rangle - \langle \nabla_U^R f,\frac{1}{1-\gamma}\pi_{>}\nabla_U^R g\rangle + \\ +&\langle \pi_{\leq} \nabla_U^L f, \Ad_{U\tilde{\gamma}^*(U_-)^{-1}} \frac{1}{1-\gamma}\pi_{>}\nabla_U^R g\rangle - \langle  \Ad_{U\tilde{\gamma}^*(U_-)^{-1}}\frac{1}{1-\gamma} \pi_{>}\nabla_U^R f,\pi_{\leq}\nabla_U^Lg\rangle.
    \end{split}
\end{equation}
Recall that $\mathcal{F}^{-1}$ is given by
\begin{equation}\label{eq:invfapndx}
\mathcal{F}^{-1}(U) = \tilde{\gamma}^*(U_-)^{-1}\cdot U, \ \ U \in \GL_n.
\end{equation}
We find it very intriguing that the maps $\mathcal{C}^{-1}$ and $\mathcal{F}^{-1}$ have very similar formulas. In a sense, $\pi_{\mathcal{F}}$ sits in between $\pi_{\std}^{\dagger}$ and $\pi_{\bg}^{\dagger}$, and it can be twisted into either of the Poisson structures via an application of $(\mathcal{F}^{-1})_*$ or $(\mathcal{C}^{-1})_*$. Is there anything interesting that one can say about $\pi_{\mathcal{F}}$, as well as about the induced compatible generalized cluster structure on $\GL_n$? 

%Although the Poisson bivector $\pi_{\mathcal{F}}$ is rational, i

\subsection{Are there cluster structures for $\mathcal{F}_m$'s?}
Let us fix a BD triple $\bg := (\Gamma_1,\Gamma_2,\gamma)$ of type $A_{n-1}$ and set
\begin{equation}\begin{split}
    \{f,g\}_{+}(U):= &\langle \pi_{>} \nabla_U^R f,\nabla_U^R g\rangle - \langle \pi_{>} \nabla_U^L f, \nabla_U^L g\rangle   +\\+ &\langle R_0\pi_0[\nabla_U f,U], [\nabla_U g,U]\rangle- \langle \pi_0 [\nabla_U f, U], \nabla_U^L g\rangle,  \ \ U \in \GL_n, 
    \end{split}
\end{equation}
where $\nabla_U^R f = U\cdot \nabla_U f$ and $\nabla_U^L f = \nabla_U f \cdot U$. Let $\hat{h}_{ij}(U):=\det U_{[i,n-j+i]}^{[j,n]}$. During a numerical experimentation\footnote{We have verified this identity in $n=3$, $n=4$ and $n=5$ for all BD triples.}, we noticed that
\[
\{\log \hat{h}_{ij},\log \hat{h}_{ks}\}_{\std}^{\dagger} = \{\log \mathcal{F}_m^*(\hat{h}_{ij}),\log\mathcal{F}_m^*(\hat{h}_{ks})\}_+ = \{\log \mathcal{F}^*(\hat{h}_{ij}),\log\mathcal{F}^*(\hat{h}_{ks})\}_{\bg}^{\dagger}
\]
for all $m \in [0,\deg \gamma]$ ($r_0$ elements are assumed to be the same). A natural question arises: does there exist a sequence of Poisson varieties\footnote{Of course, one can set $V_m$ to be the spectrum of the ring generated by the flags of $\mathcal{F}_m$. We are interested in the largest possible variety $V_m \subseteq \SL_n$ with the mentioned properties.} $(V_m,\pi_{m})$ such that $\pi_m$ reduces to $\{\cdot,\cdot\}_+$ for the flag minors of $\mathcal{F}_m$, and such that there is a generalized cluster structure $\gc_m$ on $V_m$ compatible with~$\pi_m$?

%In Proposition~\ref{p:invfk}, we showed that the rational maps $\mathcal{F}_k:\GL_n\dashrightarrow \GL_n$ have invariance properties that are similar to but weaker than the map $\mathcal{F}$.

\subsection{Are the $g$- and $h$-conventions equivalent?}\label{s:ghequiv}
By the equivalence we mean that the initial extended clusters of $\gc_h^{\dagger}(\bg)$ and $\gc_g^{\dagger}(\bg)$ can be obtained from one another via a sequence of mutations (and the variables are equal as elements of $\mathcal{O}(\GL_n)$). In \cite{multdual} we verified that the frozen variables in $\gc_g^{\dagger}(\bg,\GL_n)$ coincide with the frozen variables in $\gc_h^{\dagger}(\bg,\GL_n)$ for any BD triple $\bg$. As for the equivalence, we were able to confirm for $n=3$ and all BD triples $\bg$ that $\gc_{g}^{\dagger}(\bg,\GL_3) = \gc_{h}^{\dagger}(\bg,\GL_3)$. We conjecture that they are not equivalent for $n \geq 4$. Below we provide examples of mutation sequences that transform the initial cluster of $\gc_h^{\dagger}(\bg,\GL_3)$ into the initial cluster of $\gc_g^{\dagger}(\bg,\GL_3)$. In each case, we know all such sequences of minimal length (available upon request). Let us denote by $\varphi_{kl}^{\prime}$ and $h_{ij}^{\prime}$ the variables in the resulting extended cluster in $\gc^{\dagger}_h(\bg,\GL_3)$.

\paragraph{Case $\Gamma_1=\Gamma_2=\emptyset$.} The minimal length is $10$, the number of distinct sequences of minimal length is $8$. An example of such a sequence:
\begin{equation}
    \varphi_{12}\rightarrow\varphi_{21} \rightarrow \varphi_{11}\rightarrow h_{23} \rightarrow \varphi_{12} \rightarrow h_{23} \rightarrow \varphi_{11} \rightarrow \varphi_{21} \rightarrow h_{23} \rightarrow \varphi_{21}.
\end{equation}
The correspondence between the variables is given by $\varphi_{kl}^{\prime}(U) = \phi_{kl}(U)$ and $h_{ij}^{\prime}(U) = g_{ji}(U)$.

\paragraph{Case $\Gamma_1 = \{2\}$, $\Gamma_2 = \{1\}$.} The minimal length is $11$ and the number of sequences is $6$. An example of such a sequence:
\begin{equation}
\varphi_{12}\rightarrow \varphi_{21} \rightarrow \varphi_{11} \rightarrow h_{22}\rightarrow h_{23}\rightarrow \varphi_{12} \rightarrow h_{23} \rightarrow \varphi_{11}\rightarrow \varphi_{21}\rightarrow h_{23} \rightarrow \varphi_{21}.
\end{equation}
The correspondence between the variables is given by $\varphi_{kl}^{\prime}(U) = \phi_{kl}(U)$, $h_{23}^{\prime}(U) = g_{32}^{\prime}(U)$, $h_{22}^{\prime}(U) = g_{33}(U)$, $h_{33}(U) = g_{22}(U)$.

\paragraph{Case $\Gamma_1 = \{1\}$, $\Gamma_2 = \{2\}$.} The minimal length is $13$ and the number of sequences is $30$. An example of such a sequence:
\begin{equation}
\varphi_{12} \rightarrow h_{23} \rightarrow \varphi_{12} \rightarrow \varphi_{11} \rightarrow h_{23} \rightarrow \varphi_{21} \rightarrow \varphi_{11} \rightarrow h_{23} \rightarrow h_{33} \rightarrow \varphi_{12} \rightarrow \varphi_{11} \rightarrow \varphi_{21} \rightarrow \varphi_{11}.
\end{equation}
 %p12->h23->p12->p11->h23->p21->p11->h23->h33->p12->p11->p21->p11
The correspondence between the variables is given by $\varphi_{kl}^{\prime}(U) = \phi_{kl}(U)$, $h_{23}^{\prime}(U) = g_{32}^{\prime}(U)$, $h_{33}^{\prime}(U) = g_{22}(U)$, $h_{22}(U) = g_{33}(U)$.

\subsection{How is $\gc_h^\dagger(\bg,\SL_n^\dagger)$ related to $\gc(\bg,D(\SL_n))$?}
In the work~\cite{double}, the initial extended cluster of the generalized cluster structure $\gc_h^{\dagger}(\bg_{\std},\SL_n^{\dagger})$ was obtained from the initial extended cluster of $\gc(\bg_{\std},D(\SL_n))$ via a sequence of mutations denoted as $\mathcal{S}$. A natural question arises: if $\bg$ is any aperiodic oriented BD triple of type $A_{n-1}$, can the initial extended cluster of $\gc_h^{\dagger}(\bg,\SL_n^{\dagger})$ be obtained from the initial extended cluster of $\gc(\bg,D(\SL_n))$ that was described in~\cite{multdouble}? We found such mutation sequences\footnote{However, we didn't verify whether the sequences are of minimal possible length.} in $n=3$ and $n=4$ for all BD triples. We conjecture that the same holds for $n \geq 5$; however, we do not see a relatively simple way of proving it for an arbitrary $n$ (as one can see below, the mutation sequences become rather long and unpredictable).

Let us recall that the initial extended cluster of $\gc(\bg,D(\SL_n))$ comprises $5$ types of functions: the $g$-functions, the $h$-functions, the $\varphi$-functions, the $f$-functions and the $c$-functions. To resolve the conflict of notation, we will mark the $g$- and $h$-functions in $\gc(\bg,D(\SL_n))$ with a bar. The $\mathcal{S}$ sequence in $n=3$ is given by
\begin{equation}
    \mathcal{S}:= \bar{g}_{32}\rightarrow \bar{g}_{22}\rightarrow \bar{g}_{33}\rightarrow f_{11}\rightarrow \bar{g}_{32},
\end{equation}
and in $n=4$,
\begin{equation}
\begin{split}
    \mathcal{S}:=&\bar{g}_{42}\rightarrow \bar{g}_{32}\rightarrow \bar{g}_{43} \rightarrow \bar{g}_{22}\rightarrow \bar{g}_{33}\rightarrow \bar{g}_{44}\rightarrow f_{21}\rightarrow f_{11}\rightarrow f_{12} \rightarrow \\ &\rightarrow \bar{g}_{42}\rightarrow \bar{g}_{32}\rightarrow \bar{g}_{43} \rightarrow \bar{g}_{33} \rightarrow \bar{g}_{42}.
\end{split}
\end{equation}
Below we list the mutation sequences for $n=3$ and $n=4$, as well as the correspondence between the variables. The variables in the resulting extended cluster of $\gc(\bg,D(\SL_n))$ will be denoted as $\bar{g}^{\prime}$, $\bar{h}^\prime$ and $f^{\prime}$. The $c$- and $\varphi$-variables for $\gc(\bg,D(\SL_n))$ and $\gc_h^{\dagger}(\bg,\SL_n^{\dagger})$ are the same. The correspondence between the coordinates $(X,Y)$ in $D(\SL_n)$ and $U$ in $\SL_n$ is given by 
\[D(\SL_n)\ni (X,Y) \mapsto U:=X^{-1}Y \in \SL_n.\] Note that in the case of $D(\GL_n)$, the below correspondence between the variables is up to an additional factor of $(\det X)^{\ell}$ for some $\ell$ that depends on the given variable.

\paragraph{Case $\Gamma_1=\Gamma_2=\emptyset$, $n=3$.} The mutation sequence is given by $\mathcal{S}$. The correspondence is given by $\bar{g}_{32}^{\prime}(X,Y) = h_{33}(U)$, $f_{11}^{\prime} = h_{22}(U)$, $\bar{g}^{\prime}_{22}(X,Y) = h_{23}(U)$.

\paragraph{Case $\Gamma_1 = \{2\}$, $\Gamma_2 = \{1\}$, $n=3$.} The mutation sequence is given by
\begin{equation}
    \mathcal{S}\rightarrow \bar{h}_{12} \rightarrow \bar{h}_{22}.
\end{equation}
The correspondence is given by $\bar{h}_{22}^{\prime}(X,Y) = h_{33}(U)$, $f_{11}^\prime(X,Y) = h_{22}(U)$, $\bar{g}_{22}^{\prime}(X,Y) = h_{23}(U)$.

\paragraph{Case $\Gamma_1 = \{1\}$, $\Gamma_2 = \{2\}$, $n=3$.} The mutation sequence is given by
\begin{equation}
    \mathcal{S} \rightarrow \bar{h}_{13}\rightarrow \bar{h}_{23} \rightarrow \bar{h}_{33}\rightarrow \bar{g}_{33}\rightarrow \bar{g}_{22}\rightarrow \bar{h}_{13}\rightarrow \bar{h}_{23}\rightarrow\bar{h}_{33}.
\end{equation}
The correspondence is given by $\bar{g}_{33}^{\prime}(X,Y) = h_{23}(U)$, $\bar{h}_{33}^{\prime}(X,Y) = h_{22}(U)$, $\bar{g}_{32}^{\prime}(X,Y) = h_{33}(U)$.

\paragraph{Case $\Gamma_1 = \Gamma_2 = \emptyset$, $n=4$.} The mutation sequence is given by $\mathcal{S}$. The correspondence is given by $\bar{g}_{42}^{\prime}(X,Y) = h_{44}(U)$, $\bar{g}_{32}^{\prime}(X,Y) = h_{34}(U)$, $\bar{g}_{22}^{\prime}(X,Y) = h_{24}(U)$, $\bar{g}_{33}^{\prime}(X,Y) = h_{33}(U)$, $f_{21}^{\prime}(X,Y) = h_{23}(U)$, $f_{12}^{\prime}(X,Y) = h_{22}(U)$.

\paragraph{Case $\Gamma_1 = \{3\}$, $\Gamma_2 = \{1\}$, $n=4$.} The mutation sequence is given by
\begin{equation}
    \mathcal{S} \rightarrow \bar{h}_{12}\rightarrow \bar{h}_{22}.
\end{equation}
The correspondence is given by $\bar{h}_{22}^{\prime}(X,Y) = h_{44}(U)$, $\bar{g}_{32}^{\prime}(X,Y) = h_{34}(U)$, $\bar{g}_{22}^{\prime}(X,Y) = h_{24}(U)$, $\bar{g}_{33}^{\prime}(X,Y) = h_{33}(U)$, $f_{21}^{\prime}(X,Y) = h_{23}(U)$, $f_{12}^{\prime}(X,Y) = h_{22}(U)$.

\paragraph{Case $\Gamma_1 = \{3\}$, $\Gamma_2 = \{2\}$, $n=4$.} The mutation sequence is given by
\begin{equation}
    \mathcal{S}\rightarrow \bar{h}_{13}\rightarrow \bar{h}_{23}\rightarrow \bar{h}_{33}\rightarrow f_{11}.
\end{equation}
The correspondence is given by $f_{11}^{\prime}(X,Y) = h_{44}(U)$, $\bar{g}_{32}^{\prime}(X,Y) = h_{34}(U)$, $\bar{g}_{22}^{\prime}(X,Y) = h_{24}(U)$, $\bar{g}_{33}^{\prime}(X,Y) = h_{33}(U)$, $f_{21}^{\prime}(X,Y) = h_{23}(U)$, $f_{12}^{\prime}(X,Y) = h_{22}(U)$.

\paragraph{Case $\Gamma_1 = \{1\}$, $\Gamma_2 = \{3\}$, $n=4$.} The mutation sequence is given by
\begin{equation}
    \begin{split}
S \rightarrow &\bar{h}_{14} \rightarrow \bar{h}_{24} \rightarrow \bar{h}_{34} \rightarrow \bar{h}_{44} \rightarrow \bar{g}_{44} \rightarrow \bar{g}_{43}  \rightarrow \bar{g}_{22} \rightarrow\\ \rightarrow &\bar{h}_{14} \rightarrow \bar{h}_{24} \rightarrow \bar{h}_{34} \rightarrow \bar{h}_{44} \rightarrow \bar{g}_{44} \rightarrow \bar{g}_{22} \rightarrow f_{21} \rightarrow \\ \rightarrow & \bar{h}_{14} \rightarrow \bar{h}_{24} \rightarrow \bar{h}_{34} \rightarrow \bar{h}_{44}.
\end{split}
\end{equation}
The correspondence is given by $\bar{g}_{42}^{\prime}(X,Y) = h_{44}(U)$, $\bar{g}_{32}^{\prime}(X,Y) = h_{34}(U)$, $\bar{g}_{43}^{\prime}(X,Y) = h_{24}(U)$, $\bar{g}_{33}^{\prime}(X,Y) = h_{33}(U)$, $g_{44}^{\prime}(X,Y) = h_{23}(U)$, $h_{44}^{\prime}(X,Y) = h_{22}(U)$.

\paragraph{Case $\Gamma_1 = \{1\}$, $\Gamma_2 = \{2\}$, $n=4$.} The mutation sequence is given by
\begin{equation}
    \begin{split}
S \rightarrow &\bar{h}_{13} \rightarrow \bar{h}_{23} \rightarrow \bar{h}_{33} \rightarrow f_{11} \rightarrow \bar{g}_{22}\rightarrow \\ \rightarrow &\bar{h}_{13} \rightarrow \bar{h}_{23} \rightarrow \bar{h}_{33} \rightarrow \bar{g}_{22}\rightarrow f_{21}\rightarrow\\ \rightarrow &\bar{h}_{13} \rightarrow \bar{h}_{23}.
\end{split}
\end{equation}
The correspondence is given by $\bar{g}_{42}^{\prime}(X,Y) = h_{44}(U)$, $\bar{g}_{32}^{\prime}(X,Y) = h_{34}(U)$, $f_{11}^{\prime}(X,Y) = h_{24}(U)$, $\bar{g}_{33}^{\prime}(X,Y) = h_{33}(U)$, $\bar{h}_{33}^{\prime}(X,Y) = h_{23}(U)$, $\bar{h}_{23}^{\prime}(X,Y) = h_{22}(U)$.

\paragraph{Case $\Gamma_1 = \{2\}$, $\Gamma_2 = \{3\}$, $n=4$.} The mutation sequence is given by
\begin{equation}
    S \rightarrow \bar{h}_{14} \rightarrow \bar{h}_{24} \rightarrow \bar{h}_{34} \rightarrow \bar{h}_{44} \rightarrow \bar{g}_{44} \rightarrow \bar{g}_{43}  \rightarrow \bar{g}_{32} \rightarrow  \bar{h}_{14} \rightarrow \bar{h}_{24} \rightarrow \bar{h}_{34} \rightarrow \bar{h}_{44} \rightarrow \bar{g}_{44}.
\end{equation}
The correspondence is given by $\bar{g}_{42}^{\prime}(X,Y) = h_{44}(U)$, $\bar{g}_{43}^{\prime}(X,Y) = h_{34}(U)$, $\bar{g}_{22}^{\prime}(X,Y) = h_{24}(U)$, $\bar{g}_{44}^{\prime}(X,Y) = h_{33}(U)$, $f_{21}^{\prime}(X,Y) = h_{23}(U)$, $f_{12}^{\prime}(X,Y) = h_{22}(U)$.

\paragraph{Case $\Gamma_1 = \{2\}$, $\Gamma_2 = \{1\}$, $n=4$.} The mutation sequence is given by
\begin{equation}
    \mathcal{S}\rightarrow \bar{h}_{12}\rightarrow \bar{h}_{22}\rightarrow \bar{g}_{32}\rightarrow \bar{h}_{12}.
\end{equation}
The correspondence is given by $\bar{g}_{42}^{\prime}(X,Y) = h_{44}(U)$, $\bar{h}_{22}^{\prime}(X,Y) = h_{34}(U)$, $\bar{g}_{22}^{\prime}(X,Y) = h_{24}(U)$, $\bar{h}_{12}^{\prime}(X,Y) = h_{33}(U)$, $f_{21}^{\prime}(X,Y) = h_{23}(U)$, $f_{12}^{\prime}(X,Y) = h_{22}(U)$.

\paragraph{Case of Cremmer-Gervais, $\Gamma_1= \{2,3\}$, $\Gamma_2 = \{1,2\}$, $\gamma(i) = i-1$, $i \in \Gamma_1$.} The mutation sequence is given by
\begin{equation}
    S \rightarrow \bar{h}_{12} \rightarrow \bar{h}_{22} \rightarrow \bar{h}_{13} \rightarrow \bar{h}_{23} \rightarrow \bar{h}_{33} \rightarrow \bar{g}_{32} \rightarrow \bar{h}_{12} \rightarrow \bar{g}_{32} \rightarrow \bar{g}_{33} \rightarrow f_{11}.
\end{equation}
The correspondence is given by $f_{11}^{\prime}(X,Y) = h_{44}(U)$, $\bar{h}_{22}^{\prime}(X,Y) = h_{34}(U)$, $\bar{g}_{22}^{\prime}(X,Y) = h_{24}(U)$, $\bar{h}_{12}^{\prime}(X,Y) = h_{33}(U)$, $f_{21}^{\prime}(X,Y) = h_{23}(U)$, $f_{12}^{\prime}(X,Y) = h_{22}(U)$.

\paragraph{Case of Cremmer-Gervais, $\Gamma_1= \{1,2\}$, $\Gamma_2 = \{2,3\}$, $\gamma(i) = i+1$, $i \in \Gamma_1$.} The mutation sequence is given by
\begin{equation}\begin{split}
    S \rightarrow& \bar{h}_{13} \rightarrow \bar{h}_{23} \rightarrow \bar{h}_{33} \rightarrow \bar{h}_{14} \rightarrow \bar{h}_{24} \rightarrow \bar{h}_{34} \rightarrow \bar{h}_{44} \rightarrow f_{11}\rightarrow \bar{g}_{22}\rightarrow \bar{g}_{44} \rightarrow \bar{g}_{43} \rightarrow \bar{g}_{32} \rightarrow \\\rightarrow & \bar{h}_{13} \rightarrow \bar{h}_{23} \rightarrow \bar{h}_{33} \rightarrow \bar{h}_{14} \rightarrow \bar{h}_{24} \rightarrow \bar{h}_{34} \rightarrow \bar{h}_{44} \rightarrow \bar{g}_{44} \rightarrow f_{21} \rightarrow g_{32} \rightarrow g_{22} \rightarrow h_{14}\rightarrow\\\rightarrow& h_{24} \rightarrow h_{13}\rightarrow h_{34} \rightarrow h_{44} \rightarrow h_{23} \rightarrow g_{33}\rightarrow h_{44}\rightarrow f_{11}\rightarrow h_{33}.
    \end{split}
\end{equation}
The correspondence is given by $\bar{g}_{42}^{\prime}(X,Y) = h_{44}(U)$, $\bar{g}_{43}^{\prime}(X,Y) = h_{34}(U)$, $\bar{g}_{33}^{\prime}(X,Y) = h_{24}(U)$, $\bar{g}_{44}^{\prime}(X,Y) = h_{33}(U)$, $f_{11}^{\prime}(X,Y) = h_{23}(U)$, $\bar{h}_{33}^{\prime}(X,Y) = h_{22}(U)$.