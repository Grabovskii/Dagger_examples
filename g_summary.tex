In this section, we outline the construction of birational quasi-isomorphism for $\gc_g^{\dagger}(\bg)$, as well as the construction of the initial extended cluster. For all the other information, refer to the main paper~\cite{multdual}.

\subsection{The maps $\mathcal{F}^{\op}$, $\mathcal{Q}^{\op}$ and $\mathcal{G}^{\op}$}

\paragraph{Notation.} For a generic element $U \in \GL_n$, the element $U_+ \in \GL_n$ is a unipotent upper triangular matrix and $U_\ominus\in \GL_n$ is a lower triangular matrix, such that $U = U_+ U_\ominus$.

\paragraph{The map $\mathcal{F}^{\op}$.}  Let $\bg := (\Gamma_1,\Gamma_2,\gamma)$ be a BD triple of type $A_{n-1}$. Define the sequence $\mathcal{F}_k^{\op}:\GL_n\dashrightarrow \GL_n$ of rational maps via
\begin{equation}
    \mathcal{F}_0^{\op}(U):= U, \ \ \mathcal{F}_k^{\op}(U):= U\tilde{\gamma}[\mathcal{F}_{k-1}^{\op}(U)_+], \ \ k \geq 1.
\end{equation}
The birational map $\mathcal{F}^{\op}:\GL_n \dashrightarrow \GL_n$ is defined as the limit 
\begin{equation}\label{eq:def_fop}
    \mathcal{F}^{\op}(U):= \lim_{k \rightarrow \infty} \mathcal{F}_k^{\op}(U).
\end{equation}
Since $\gamma$ is nilpotent, the sequence $\mathcal{F}_k^{\op}$ stabilizes at $k = \deg \gamma$, so $\mathcal{F}^{\op}(U) = \mathcal{F}_{\deg \gamma}^{\op}(U)$. The inverse of $\mathcal{F}^{\op}$ is given by
\begin{equation}
    (\mathcal{F}^{\op})^{-1}(U):= U\tilde{\gamma}(U_+)^{-1}.
\end{equation}
The map $\mathcal{F}^{\op}$ is neither a Poisson map nor a quasi-isomorphism. However, by means of $\mathcal{F}^{\op}$ one can construct Poisson birational quasi-isomorphisms in the $g$-convention. For various invariance properties of $\mathcal{F}^{\op}$, refer to \cite[Section 7.1]{multdual}.

\paragraph{Birational quasi-isomorphisms.} Define the birational map $\mathcal{Q}^{\op}:\GL_n \dashrightarrow \GL_n$ via
\begin{equation}
    \mathcal{Q}^{\op}(U):= \rho^{\op}(U) U (\rho^{\op}(U))^{-1}, \ \ \rho^{\op}(U):=\prod_{i=1}^{\leftarrow}[\tilde{\gamma}]^{i}(U_+).
\end{equation}
The inverse of $\mathcal{Q}^{\op}$ is given by the map
\begin{equation}
    (\mathcal{Q}^{\op})^{-1}(U):=\mathcal{F}^{\op,c}(U):= \tilde{\gamma}(\mathcal{F}^{\op}(U)_+)^{-1}\mathcal{F}^{\op}(U).
\end{equation}
Let $\pi_{\bg}^{\dagger}$ and $\pi_{\std}^{\dagger}$ be the Poisson bivectors associated with an arbitrary BD triple $\bg$ and $\bg_{\std}$ (of type $A_{n-1}$), respectively. If the $r_0$ parts of $\pi_{\bg}^{\dagger}$ and $\pi_{\std}^{\dagger}$ are the same, then $\mathcal{Q}^{\op}:(\GL_n,\pi_{\std}^{\dagger}) \dashrightarrow (\GL_n,\pi_{\bg}^{\dagger})$ is a Poisson isomorphism. Moreover, as a map $\mathcal{Q}^{\op}:(\GL_n,\gc_g^{\dagger}(\bg_\std))\dashrightarrow (\GL_n,\gc_g^{\dagger}(\bg))$, it is a birational quasi-isomorphism, with the marked variables given by 
\begin{equation}
    \{g_{i+1,i+1} \ | \ i \in \Gamma_1\}.
\end{equation}

If $\tilde{\bg}\prec \bg$ is another BD triple of type $A_{n-1}$, then there is a birational quasi-isomorphism $\mathcal{G}^{\op}:(\GL_n,\gc_h^{\dagger}(\tilde{\bg})) \dashrightarrow (\GL_n,\gc_h^{\dagger}(\bg))$. If $\tilde{\mathcal{Q}}^{\op}$ is defined as the map $\mathcal{Q}^{\op}$, but with respect to the BD triple $\tilde{\bg}$, then $\mathcal{G}^{\op} = \mathcal{Q}^{\op}\circ \tilde{\mathcal{Q}}^{\op}$. As a map $\mathcal{G}^{\op}:(\GL_n,\pi_{\tilde{\bg}}^{\dagger})\dashrightarrow (\GL_n,\pi_{\bg}^{\dagger})$, it is a Poisson isomorphism if the $r_0$ parts of $\pi_{\tilde{\bg}}^{\dagger}$ and $\pi_{\bg}^{\dagger}$ are the same. The marked variables for $\mathcal{G}^{\op}$ are given by
\begin{equation}
    \{g_{i+1,i+1} \ | \ i \in \Gamma_1\setminus \tilde{\Gamma}_1\}.
\end{equation}
Explicit formulas for $\mathcal{G}^{\op}$ can be obtained from explicit formulas for $\mathcal{G}$ (refer to \cite[Section 4.4, Section 4.5, Section 7.3]{multdual}).

\subsection{Initial extended cluster} The initial extended cluster comprises three types of functions: $c$-functions, $\phi$-functions and $g$-functions. Only the description of the $g$-functions depends on the choice of the Belavin-Drinfeld triple.

\paragraph{Description of $\phi$- and $c$-functions.}For an element $U \in \GL_n$, let us set
\begin{equation}\label{eq:big_phi_def_g}
\Phi_{kl}^\prime(U):=\begin{bmatrix}(U^0)^{[1,k]} & U^{[1,l]} & (U^2)^{\{1\}} & \cdots & (U^{n-k-l+1})^{\{1\}}\end{bmatrix}, \ \ k,l \geq 1, \ k+l \leq n;
\end{equation}
\begin{equation}\label{eq:s_def2}
s_{kl}:=\begin{cases}
(-1)^{k(l+1)} \ &n \ \text{is even},\\
(-1)^{(n-1)/2 + k(k-1)/2 + l(l-1)/2} \ & n \ \text{is odd}.
\end{cases}
\end{equation}
Then the $\phi$-functions are given by
\begin{equation}\label{eq:phi_g_def}
\phi_{kl}(U):=s_{kl} \det \Phi_{kl}^\prime(U)
\end{equation}
The $c$-functions are uniquely defined via
\begin{equation}\label{eq:c_def}
\det(I+\lambda U) = \sum_{i=0}^{n} \lambda^{i} s_i c_i(U)
\end{equation}
where $s_i := (-1)^{i(n-1)}$ and $I$ is the identity matrix. Note that $c_0 = I$ and $c_n = \det U$ (the $c$-functions are the same in both $g$- and $h$-conventions).

\paragraph{Description of the $g$-functions.} Let $\Pi$ be a set of simple roots of type $A_{n-1}$ and let $\bg:=(\Gamma_1,\Gamma_2,\gamma)$ be a BD triple of type $A_{n-1}$. Let $\mathcal{F}^{\op}:\GL_n \dashrightarrow \GL_n$ be the rational map defined by~\eqref{eq:def_fop}. We identify $\Pi$ with the interval $[1,n-1]$. For a given $\alpha_0 \in \Pi \setminus \Gamma_1$, set $\alpha_t:=\gamma^*(\alpha_{t-1})$, $t \geq 1$. Recall that the sequence $S^{\gamma^*}(\alpha_0):=\{\alpha_{t}\}_{t \geq 0}$ is the \emph{$\gamma^*$-string} associated to $\alpha_0$; $\gamma^*$-strings partition $\Pi$. For each $\alpha_0 \in \Pi\setminus\Gamma_1$ and the associated $\gamma^*$-string $S^{\gamma^*}(\alpha_0):= \{\alpha_i\}_{i=0}^m$, for every $k \in [0,m]$ and $i \in [\alpha_k+1,n]$, define
\begin{equation}\label{eq:g_fun}
    g_{i,\alpha_k+1}(U) := \det [\mathcal{F}^{\op}(U)]_{[i,n]}^{[\alpha_k+1,n-i+\alpha_k+1]} \prod_{t \geq k+1}^{m} \det[\mathcal{F}^{\op}(U)]_{[\alpha_t+1,n]}^{[\alpha_t+1,n]}.
\end{equation}
We refer to the functions $g_{ij}$, $2 \leq j \leq i \leq n$, together with $g_{11}(U):=\det U$ as the \emph{$g$-functions}. 

\paragraph{Frozen variables.} In the case of $\gc_g^{\dagger}(\bg,\GL_n)$, the frozen variables are given by the set
\begin{equation}
\{c_1,c_2,\ldots,c_{n-1}\} \cup \{g_{i+1,i+1} \ | \ i \in \Pi \setminus \Gamma_1\} \cup \{g_{11}\}.
\end{equation}
In the case of $\gc_h^{\dagger}(\bg,\SL_n)$, $g_{11}(U) = 1$, so this variable is absent. The zero loci of the frozen variables foliate into unions of symplectic leaves of the ambient Poisson variety $(\GL_n,\pi_{\bg}^{\dagger})$ or $(\SL_n,\pi_{\bg}^{\dagger})$. Moreover, the frozen $h$-variables do not vanish on $\SL_n^{\dagger}$.

\paragraph{Initial extended cluster.} The initial extended cluster $\Psi_0$ of $\gc_g^{\dagger}(\bg,\GL_n)$ is given by the set
\begin{equation}\label{eq:iniext}
    \{g_{ij} \ | \ 2 \leq j \leq i \leq n\} \cup \{\phi_{kl} \ | \ k,l \geq 1, \ k+l\leq n\} \cup \{c_1,\ldots,c_{n-1}\}\cup\{g_{11}\}.
\end{equation}
The initial extended cluster of $\gc_g^{\dagger}(\bg,\SL_n)$ is obtained from $\Psi_0$ via removing $h_{11}$.

\paragraph{A generalized cluster mutation.} In the initial extended cluster, only the variable $\phi_{11}$ is equipped with a nontrivial \emph{string}, which is given by $(1,c_1,\ldots,c_{n-1},1)$. The generalized mutation relation for $\phi_{11}$ reads
\begin{equation}\label{eq:p11mut}
\phi_{11} \phi_{11}^{\prime} = \sum_{r=0}^n c_r \phi_{21}^{r} \phi_{12}^{n-r}.
\end{equation}
Other mutations of the initial extended cluster follow the usual pattern from the theory of cluster algebras of geometric type.