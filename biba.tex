\begin{thebibliography}{99}
%\bibitem{bd} A. Belavin and V. Drinfeld, 'Solutions of the classical Yang-Baxter equation for simple Lie algbras', \emph{Funktsional. Anal. i Prilozhen} \textbf{16}(1982), 159--180. \doi{10.1007/BF01081585}

%\bibitem{bd2} A. Belavin and V. Drinfeld, \emph{{T}riangle {E}quations and {S}imple {L}ie {A}lgebras} (Hardwood Academic, 1998).

%\bibitem{upper_bounds} A. Berenstein, S. Fomin and A. Zelevinsky, 'Cluster algebras III: Upper bounds and double Bruhat cells', \emph{Duke Math. J.} \textbf{126}(1) (2005), 1--52. \doi{10.1215/S0012-7094-04-12611-9}

%\bibitem{bfz_param} A. Berenstein, S. Fomin and A. Zelevinsky, 'Parametrizations of canonical bases and totally positive matrices', \emph{Adv. Math.} \textbf{122}(1) (1996), 49--149. \doi{10.1006/aima.1996.0057}

% \bibitem{berenstein2005quantum} A. Berenstein and A. Zelevinsky, 'Quantum cluster algebras', \emph{Adv. Math.} \textbf{195}(2005), 405--455.\\ \doi{10.1016/j.aim.2004.08.003}

%\bibitem{dual_brahami}R. Brahami, 'Cluster $\Chi$-varieties for dual Poisson-Lie groups, I', \emph{Algebra i Analiz} \textbf{22}(2) (2010), 14--104.
%\doi{10.1090/S1061-0022-2011-01138-0}

%\bibitem{chari} V. Chari and A. Pressley, \emph{A {G}uide to {Q}uantum {G}roups} (Cambridge Univ. Press, 1995).

%  \bibitem{earlier} L. Chekhov and M. Shapiro, 'Teichm{\"u}ller spaces of Riemann surfaces with orbifold points of arbitrary order and cluster variables' \emph{Int. Math. Res. Not. IMRN} \textbf{10} (2014), 2746--2772.\\ \doi{10.1093/imrn/rnt016}

%\bibitem{bdr} P. Delorme, 'Classification des triples de Manin pour les algebres de Lie reductives complexes: Avec un appendice de Guillaume Macey', \emph{J. Algebra} \textbf{246}(1) (2001), 97--174.\\ \doi{10.1006/jabr.2001.8887}

%\bibitem{slfive} I. Eisner, 'Exotic cluster structures on $\SL_5$', \emph{J. Phys. A} \textbf{47}(2014) 474002.\\ {\doi{10.1088/1751-8113/47/47/474002}}

% \bibitem{etingof} P. Etingof and O. Schiffmann, \emph{Lectures on {Q}uantum {G}roups} (International Press, 1998).

%\bibitem{etingofdyn} P. Etingof, T. Schedler, O. Schiffmann, 'Explicit quantization of dynamical r-matrices for finite dimensional semisimple Lie algebras', \emph{J. Amer. Math. Soc.} \textbf{13}(3) (2000), 595--609. \\ \doi{10.1090/S0894-0347-00-00333-7}

%\bibitem{fockcoords} V. Fock and A. Goncharov, 'Moduli spaces of local systems and higher Teichm\"uller theory', \emph{Publ. Math. Inst. Hautes \'Etudes Sci.} \textbf{103}(2006), 1--211.

%  \bibitem{fock2006cluster}
% V. Fock and A. Goncharov, 'Cluster $\chi$-varieties, amalgamation, and Poisson-Lie groups', \emph{Progr. Math.} \textbf{253}(2006), 27--68. \doi{10.1007/978-0-8176-4532-8_2}

% \bibitem{tensordiags} S. Fomin and P. Pylyavskyy, 'Tensor diagrams and cluster algebras', \emph{Adv. Math.} \textbf{300}(10) (2016), 717--787. \\ \doi{10.1016/j.aim.2016.03.030}

% \bibitem{fraser} C. Fraser, 'Quasi-Homomorphisms of Cluster Algebras', \emph{Adv. in Appl. Math.} \textbf{81}(2016), 40--77. \\\doi{10.1016/j.aam.2016.06.005}

%\bibitem{fomin6} S. Fomin, L. Williams and A. Zelevinsky, 'Introduction to Cluster algebras. Chapter 6', Preprint, 2020. \href{https://arxiv.org/abs/2008.09189}{arXiv:2008.09189}.
%\href{https://arxiv.org/abs/2008.09189}{\nolinkurl{arXiv:2008.09189}}. %<--- makes it bold and not beautiful

%\bibitem{fathers} S. Fomin and A. Zelevinsky, 'Cluster algebras I: Foundations', \emph{J. Amer. Math. Soc.} \textbf{15}(2) (2002), 497--529. \doi{10.1090/S0894-0347-01-00385-X}

%\bibitem{gantmacher} F. Gantmacher, \emph{The theory of matrices, vol. 1} (Amer. Math. Soc., Providence, RI, 1998)
%\bibitem{chari} V. Chari and A. Pressley, \emph{A {G}uide to {Q}uantum {G}roups} (Cambridge Univ. Press, 1995).

%\bibitem{rho} M. Gekhtman, M. Shapiro and A. Vainshtein, 'A unified approach to exotic cluster structures on simple Lie groups', Preprint, 2023. \href{https://arxiv.org/abs/2308.16701}{arXiv:2308.16701}.

%\href{https://arxiv.org/abs/2308.16701}{\nolinkurl{arXiv:2308.16701}}.

 %\bibitem{dasbuch} M. Gekhtman, M. Shapiro and A. Vainshtein, '{C}luster algebras and {P}oisson geometry', \emph{Math. Surveys Monogr.} \textbf{167}(2010). \doi{10.1090/surv/167}

%\bibitem{roots} M. Gekhtman, M. Shapiro and A. Vainshtein, 'Cluster algebras and Poisson geometry', \emph{Mosc. Math. J.} \textbf{3}(3) (2003), 899-934. \doi{10.17323/1609-4514-2003-3-3-899-934}; \href{https://arxiv.org/abs/math/0208033}{arXiv:0208033}.

%\bibitem{conj} M. Gekhtman, M. Shapiro and A. Vainshtein, 'Cluster structures on simple complex Lie groups and Belavin-Drinfeld classification', \emph{Mosc. Math. J.} \textbf{12}(2) (2010), 293--312. 
%% %\doi{10.17323/1609-4514-2012-12-2-293-312} %the doi is broken

% \bibitem{exotic}  M. Gekhtman, M. Shapiro, and A Vainshtein, 'Exotic cluster structures on $\SL_n$: the {C}remmer-{G}ervais case', \emph{Mem. Amer. Math. Soc.} \textbf{246}(1165) (2017), 1--94. \doi{10.1090/memo/1165}

\bibitem{double} M. Gekhtman, M. Shapiro and A. Vainshtein, 'Drinfeld double of $\GL_n$ and generalized cluster structures', \emph{Proc. Lond. Math. Soc.} \textbf{116}(3) (2018), 429--484. \doi{10.1112/plms.12086}

%\bibitem{doublerel} M. Gekhtman, M. Shapiro and A. Vainshtein, 'Generalized cluster structures related to the Drinfeld double of $\GL_n$', \emph{J. Lond. Math. Soc.} \textbf{105}(3), (2022),
%  1601--1633.\\ \doi{10.1112/jlms.12542}

%\bibitem{periodic} M. Gekhtman, M. Shapiro and A. Vainshtein, 'Periodic staircase matrices and generalized cluster structures', \emph{Int. Math. Res. Not. IMRN} \textbf{2022}(6) (2022), 4181--4221.\\ \doi{10.1093/imrn/rnaa148}

%\bibitem{plethora} M. Gekhtman, M. Shapiro and A. Vainshtein, 'Plethora of cluster structures on $\GL_n$', Preprint, 2019. \href{https://arxiv.org/abs/1902.02902}{arXiv:1902.02902}.

%\bibitem{grass_poiss} M. Gekhtman, M. Shapiro and A. Vainshtein, 'Poisson geometry of directed networks in a disk', \emph{Selecta Math. (N.S.)} \textbf{15} (2009), 61--103. \doi{10.1007/s00029-009-0523-z}

%\bibitem{yakimov_locus} K. R. Goodearl, M. T. Yakimov, 'Cluster algebra structures on Poisson nilpotent algebras', Preprint, 2018. \doi{10.48550/arXiv.1801.01963}

%\bibitem{ivan_ip} I. Ip, 'Cluster realization of $U_q(\mathfrak{g})$ and factorizations of the universal $R$-matrix, \emph{Selecta Math.} \textbf{24}(5) (2018), 4461--4553. \doi{10.1007/s00029-018-0432-0}

%\bibitem{hodges} T. J. Hodges, 'On the Cremmer-Gervais quantizations of $\SL(n)$', \emph{Int. Math. Res. Not. IMRN} \textbf{10} (1995), 465--481. \doi{10.1155/S107379289500033X}

%\bibitem{open_rich} B. Leclerc, 'Cluster structures on strata of flag varieties', \emph{Adv. Math.} \textbf{300}(10) (2016), 190--228.\\ \doi{10.1016/j.aim.2016.03.018}

%\bibitem{yakimov_det} B. Nguyen, K. Trampel and M. Yakimov, 'Noncommutative discriminants via Poisson primes', \emph{Adv. Math.} \textbf{322}(2017), 269--307. \doi{10.1016/j.aim.2017.10.018}

%\bibitem{rs}  A. Reyman and M. Semenov-Tian-Shansky, \emph{{I}ntegrable systems.} (Institute of Computer Studies, Moscow, 2003).

%\bibitem{serre} J.-P. Serre, \emph{Complex semisimple Lie algebras.} (Springer Science \& Business Media, New York, 1987).

%  \bibitem{schneider} H. Schneider, 'The concepts of irreducibility and full indecomposability of a matrix in the works of Frobenius, K{\"o}nig and Markov', \emph{Linear Algebra Appl.} \textbf{18}(12) (1977), 139--162. \doi{10.1016/0024-3795(77)90070-2} 

\bibitem{multdouble} D. Voloshyn, 'Multiple generalized cluster structures on $D(\mathrm{GL}_n)$', \emph{Forum of Mathematics, Sigma} \textbf{11}(46) (2023), 1--78. \doi{10.1017/fms.2023.44}

%\bibitem{zametka} D. Voloshyn, 'Starfish lemma via birational quasi-isomorphisms', Preprint, 2023. \\ \href{https://arxiv.org/abs/2311.00404}{arXiv:2311.00404}.

 \bibitem{multdual} D. Voloshyn and M. Gekhtman, 'Generalized cluster structures on $\SL_n^{\dagger}$', Preprint, 2023. \href{https://arxiv.org/abs/2312.04859}{arXiv:2312.04859}  

%\bibitem{sasha_group} G. Schrader and A. Shapiro, 'A cluster realization of $U_q(\sll_n)$ from quantum character varieties', \emph{Invent. Math} \textbf{216}(2019), 799--846. \doi{10.1007/s00222-019-00857-6}

%\bibitem{sasha_gauge} G. Schrader and A. Shapiro, '$K$-theoretic Coulomb branches of quiver gauge theories and cluster varieties', Preprint, 2019. \href{https://arxiv.org/abs/1910.03186}{arXiv:1910.03186}.

%\bibitem{scott} J. Scott, 'Grassmannians and cluster algebras', \emph{Proc. Lond. Math. Soc.} \textbf{92} (2006), 345--380. \\ \doi{10.1112/S0024611505015571}

%\bibitem{bott_shen} L. Shen and D. Weng, 'Cluster structures on double Bott-Samelson cells', \emph{Forum of Mathematics, Sigma} \textbf{9}(66) (2021), 1-89. \doi{10.1017/fms.2021.59}

%\bibitem{dual_shen} L. Shen, 'Duals of semisimple Poisson-Lie groups and cluster theory of moduli spaces of $G$-local systems', \emph{Int. Math. Res. Not. IMRN} \textbf{2022}(18), 14295--14318. \doi{10.1093/imrn/rnab094}

%1910.03186

\end{thebibliography}